\documentclass[11pt, spanish]{report}
\usepackage[spanish]{babel}
\selectlanguage{spanish}
\usepackage[utf8]{inputenc}
\usepackage{amsmath}
\usepackage{amsfonts}
\usepackage{amsthm}
\usepackage{listings}

% listings de C (pseudocódigo en realidad).
\lstset{language=C,frame=single,numbers=left,morekeywords={foreach}}

% Tikz
\usepackage{pgf}
\usepackage{tikz}
\usetikzlibrary{arrows,automata}

% Margenes
\usepackage[inner=3cm,outer=2cm]{geometry}
\setlength{\oddsidemargin}{5.5pt}
\setlength{\evensidemargin}{5.5pt}
%Espaciado
%\linespread{1.3}

\newcommand{\transition}{l}
\newcommand{\globally}{\square}
\newcommand{\finally}{\lozenge}

\newcommand{\readf}{\texttt{read()\,}}
\newcommand{\writef}{\texttt{write()\,}}

\title{Sistemas Operativos}
\date{}

\begin{document}
\maketitle

\chapter{Sistemas Distribuidos}

Conjunto de recursos conectados que interactúan.
\begin{itemize}
  \item Varias máquinas conectadas en red.
  \item Un procesador con varias memorias.
  \item Varios procesadores que comparen una (o más) memoria(s).
\end{itemize}

\begin{itemize}
  \item Fortalezas \begin{itemize}
      \item Paralelismo.
      \item Replicación.
      \item Descentralización.
    \end{itemize}
  \item Debilidades \begin{itemize}
      \item Sincronización.
      \item Coherencia.
      \item \textbf{Información parcial.}
    \end{itemize}
\end{itemize}

\section{Memoria compartida}

Hardware
\begin{itemize}
\item Uniform Memory Access (UMA) Procesadores 
\end{itemize}


\section{Sincronización}

Modelos de comunicación.
\begin{itemize}
  \item Envío asincrónico de mensajes.
  \item Memoria compartida global direccionable.
\end{itemize}

Problemas.
\begin{itemize}
  \item Orden de ocurrencia de los eventos.
  \item Exclusión mutua.
  \item Consenso.
\end{itemize}

\subsection{Modelo de proceso}

\begin{tikzpicture}[align=center,node distance=2.8cm]
  \tikzstyle{every state}=[fill=red,draw=none,text=white]

  \node[initial,state] (R)              {$R$};
  \node[state]         (T) [right of=R] {$T$};
  \node[state]         (C) [right of=T] {$C$};
  \node[state]         (E) [right of=C] {$E$};

  \path (R) edge              node {try} (T)
        (T) edge              node {crit} (C)
        (C) edge              node {exit} (E)
        (E) edge [bend right]  node {rem} (R);
\end{tikzpicture}




\begin{itemize}
  \item $N$ procesos.
  \item Estado: $\sigma : \{0,\dots,N-1\} \to \{R, T, C, E\}$.
  \item Transición: $\sigma \xrightarrow{\transition} \sigma'$, $\transition \in \{rem, try, crit, exit\}$.
  \item Ejecución: $\tau = \tau_0 \xrightarrow{\transition} \tau_1 \xrightarrow{\transition'} \dots$.
  \item $\globally$ significa globalmente, siempre.
  \item $\finally$ significa finalmente, eventualmente.
\end{itemize}

\textbf{Exclusión mutua (EXCL)}
Para toda ejecución $\tau$ y estado $\tau_k$, no puede haber más de un proceso $i$ tal que $\tau_k(i) = C$.
\[\globally \#CRIT \leq 1\]

\textbf{Progreso (PROG)} (lock-free)
Para toda ejecución $\tau$, estado $\tau_k$ y todo proceso $i$, si $\tau_k(i) = T$, entonces $\exists j > k$, tal que $\tau_j(i) = C$.

\[IN(i) \equiv i\in TRY \implies \finally i \in CRIT\]
\[\forall i . \globally IN(i)\]

\textbf{Prograso global dependiente (G-PROG)} (deadlock-free, lockout-free, starvation-free)
Para toda ejecución $\tau$,
si para todo estado $\tau_k$ y proceso $i$ tal que $\tau_k(i) = C$,
  $\exists j > k$, tal que $\tau_j(i) = R$.
entonces para todo estado $\tau_{k'}$ y todo proceso $i'$,
  si $\tau_{k'}(i') = T$
  entonces $\exists j' > k'$, tal que $\tau_{j'}(i') = C$.

\[OUT(i) \equiv i \in CRIT \implies \finally i \in REM\]
\[(\forall \globally i. OUT(i)) \implies (\forall i . \globally IN(i))\]

\textbf{Justicia (FAIR)} (fairness)
Para toda ejecución $\tau$ y todo proceso $i$,
si $i$ puede hacer una transición $\transition_i$ en una cantidad infinita de estados de $\tau$
entonces existe un $k$ tal que $\tau_k \xrightarrow{\transition_i} \tau_{k+1}$.

\subsection{Registros RW}
Objeto atómico básico: \textbf{read-write register}.

Procesos por operación: \textbf{single/multiple}.

\begin{itemize}
  \item Si \readf y \writef no se solapan, \readf devuelve el \textbf{último} valor escrito.
  \item Si \readf y \writef se solapan \begin{itemize}
      \item ``Safe'': \readf devuelve \textbf{cualquier} valor.
      \item Regular: \readf devuelve \textbf{algún} valor escrito.
      \item \textbf{Atomic}: \readf devuelve un valor \textbf{consistente} con una serialización.
    \end{itemize}
\end{itemize}

\subsubsection{EXCL con registros RW: Dijkstra}
Registros:
\begin{itemize}
  \item \texttt{flag[i]}: atomic single-writer / multi-reader.
  \item \texttt{turn}: atomic multi-writer / multi-reader.
\end{itemize}

Proceso $i$:
\begin{lstlisting}
  /* Try */
  L:
    flag[i] = 1;
    while (turn != i) {
      if (flag[turn] == 0) turn = i;
    }
    flag[i] = 2;
    foreach (j != i) {
      if (flag[j] == 2) goto L;
    }
  /* Crit */
  ...
  /* Exit */
  flag[i] = 0;
\end{lstlisting}

\begin{itemize}
  \item Garantiza EXCL.
  \item Asumiendo FAIR, garantiza PROG, pero no G-PROG.
\end{itemize}

La idea de la demostración de que, asumiendo FAIR garantiza PROG es por el absurdo.
Supongamos que hay una cadena de pasos $\alpha$ que es FAIR pero que no satisface PROG, es decir, eventualmente hay alguien en $T$ y nadie en $C$, pero nunca más nadie entra en $C$.
Veamos que le pasa a $\alpha$.
\begin{enumerate}
  \item Eventualmente, todos los procesos, o bien van a estar en $T$ o bien van a estar en $R$ (pues como es fair, si estan en $E$ eventualmente van a llegar a $R$ o $T$.
  \item Eventualmente esos procesos no van a cambiar nunca más de estado (es decir, no va a haber mas transiciones de $R$ a $T$. Además, para todo $i$ en $T$, $\texttt{flags[i]} \leq 1$.
  \item Eventualmente, $\texttt{turn}$ va a valer $i$ para algún $i$ en $T$ y no va a cambiar (es decir, $\texttt{turn}$ se estabiliza).
  \item Eventualmente, todos los $j \neq i$ van a tener $\texttt{flags[j]} \neq 2$, pues no pueden pasar el primer loop. Luego, el proceso $i$ entra a $C$, absurdo.
\end{enumerate}


\subsubsection{EXCL con registros RW: Panadería de Lamport}
Registros:
\begin{itemize}
  \item \texttt{choosing[i]}, \texttt{number[i]}: atomic single-writer / multi-reader.
\end{itemize}

Proceso $i$:
\begin{lstlisting}
  /* Try */
  choosing[i] = 1;
  number[i] = 1 + max {number[j] | j != i};
  choosing[i] = 0;
  foreach j != i {
    waitfor choosing[j] == 0;
    waitfor number[j] == 0 || (number[i], i) < (number[j], j);
  }
  /* Crit */
  ...
  /* Exit */
  number[i] = 0;
\end{lstlisting}

\begin{itemize}
  \item Garantiza EXCL, PROG y G-PROG.
\end{itemize}

La idea de la demostración de que garantiza PROG es por el absurdo y es muy fácil. Igual que antes, eventualmente todos van a estar en $T$ o $R$, sin pasarse. De los que están en $T$, alguno va a tener el menor par $(\texttt{number[i]}, i)$, y ese va a pasar a $C$.







\end{document}
